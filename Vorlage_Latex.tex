\documentclass[paper=a4,
fontsize=12pt,
toc=listof,
bibliography=totoc,
numbers=noendperiod %Kein Punkt am Ende von Kapitelzahlen bei Mischung von arabisch und buchstaben im Verzeichnis
]{scrreprt}

\usepackage[utf8]{inputenc}
\usepackage[T1]{fontenc}
\usepackage[ngerman]{babel}

\usepackage{acronym}

\usepackage{helvet}
\usepackage{txfonts}



\usepackage{graphicx}
\usepackage{csquotes} 

\usepackage{pdfpages}

\usepackage{paralist}
\usepackage{hhline}
\usepackage{colortbl}
\usepackage{tabularx}
\usepackage{longtable}
\usepackage{multirow}



\usepackage[headsepline]{scrlayer-scrpage}

%Schriftgröße der Fußnote auf 10pt, Fett und Trennung per Doppelpunkt, Abstand zu Grafik 5pt
\usepackage[margin=0pt,font={footnotesize, bf},labelfont=bf, labelsep=colon, skip=3pt]{caption}

%Abstand unterhalb Fließumgebungen 3pt
\setlength{\intextsep}{3pt}
%
%\makeatletter
%\renewcommand\normalsize{\@setfontsize\normalsize{12}{12}}
%\renewcommand\large{\@setfontsize\large{14}{16.8}}
%\renewcommand\LARGE{\@setfontsize\footnotesize{10}{11.2}}
%\renewcommand\huge{\@setfontsize\small{10}{11.2}}
%\makeatother

%Durch ausgewählte Schriftgröße is baselineskip auf 18pt
%Der aktuellen Word-Vorlage entsprechend sind die Abstände Vor/Nach Überschriftena auf vor 6pt, nach 12pt

\RedeclareSectionCommand[
  %runin=false,
  afterindent=false,
  beforeskip=0.33\baselineskip,
  afterskip=0.66\baselineskip,
  ]{chapter}
\RedeclareSectionCommand[
  %runin=false,
%  style=
  afterindent=false,
  beforeskip=0.33\baselineskip,
  afterskip=0.66\baselineskip
  ]{section}
\RedeclareSectionCommand[
  %runin=false,
  afterindent=false,
  beforeskip=0.33\baselineskip,
  afterskip=0.66\baselineskip
  ]{subsection}
\RedeclareSectionCommand[
  %runin=false,
  afterindent=false,
    beforeskip=0.33\baselineskip,
  afterskip=0.66\baselineskip
  ]{subsubsection}
\RedeclareSectionCommand[
  runin=true,
  %afterindent=false,
  beforeskip=.15\baselineskip,
  afterskip=.15\baselineskip]{paragraph}
\RedeclareSectionCommand[
  runin=true,
  %afterindent=false,
  beforeskip=.15\baselineskip,
  afterskip=.15\baselineskip]{subparagraph}


%Durch Setzen der Setzen der 	Standardschriftgröße auf 12pt gelten die folgenden Schriftgrößen der Abstufungen:
%\tiny	6pt
%\scriptsize	8pt
%\footnotesize	10pt
%\small	11pt
%\normalsize	12pt
%\large	14pt
%\Large	17pt
%\LARGE	20pt
%\huge	25pt
%\Huge		25pt

\parindent=0pt

\setkomafont{chapter}{\large}
\setkomafont{section}{\normalsize}
\setkomafont{subsection}{\normalsize}

\setlength{\headheight}{1.5\baselineskip}
\pagestyle{scrheadings}
\clearscrheadfoot
\ihead{\headmark} %Aktuellen Kolumnennamen Kopfzeile links
\automark{chapter} %Definition, dass nur Kapitelname in Kopfzeile soll
\ohead{\pagemark}
\renewcommand{\chapterpagestyle}{scrheadings}



\newcolumntype{Y}{>{\centering\arraybackslash}X}

% Da biblatex verwendet wird, bietet es sich an im Übersetzungsprozess
% biber zu verwenden
\usepackage[style=authoryear, backend=biber, bibencoding=utf8, maxbibnames=99]{biblatex}
\DeclareDelimFormat{finalnamedelim}{\addspace\&\addspace}
\DefineBibliographyStrings{ngerman}{andothers={et\addabbrvspace al\adddot}}





\usepackage[left=30mm,right=25mm,top=25mm,bottom=15mm, headsep=5mm]{geometry}
\usepackage[onehalfspacing]{setspace}


%Hidelinks damit keine roten Kästen in PDF sind, Breaklinks für automatischen Umbruch von links, insb. im Literaturverzeichnis
\usepackage[hidelinks , breaklinks= true]{hyperref}


\newcounter{savepage}




\addto\captionsngerman{
  \renewcommand{\contentsname}
    {Gliederung}
}

\addbibresource{Quellen/Quellen.bib}

\begin{document}
\begin{titlepage}
{{ \noindent  Universität Leipzig \\
		Wirtschaftswissenschaftliche Fakultät\\
		\itshape Prof. Dr. Rainer Alt\\
		Finn Trygve Jessen\\	 \par}
	\vspace{2cm}
	{\centering  Thema \\
	\bfseries \large Marktüberblick über bestehende Ansätze und Entwicklung von Anforderungen für einen DLT-Datenmarktplatz\par}
	\vspace{2cm}
	{ \centering  Masterarbeit zur Erlangung des akademischen Grades\\
	     Master of Science – Wirtschaftsinformatik
		\par}

	\vfill}

	{\noindent Vorgelegt von: \textit{Platz,  Halter } \\
	 Matrikelnummer: \textit{XXXXXXX} \\
	 E-Mail-Adresse: \textit{lXXXX@XXXX.de} \\
	 Telefonnummer: \textit{XXXXXX} \\
	 Anschrift:  {XXXXXXX X\\
						\hspace*{1.75cm} 04161 Leipzig} \\ } 
		
		{\noindent  Leipzig, den Abgabedatum\par}
\end{titlepage}

\pagenumbering{Roman}

%Abstract
\addchap{Abstract}
In dieser Arbeit wird untersucht, ob die Distributed-Ledger-Technologie (DLT) sich zur Verwendung als Trägertechnologie für einen Datenmarkt eignet und so domänenübergreifende Datennutzung ermöglicht werden kann.  Ebenso wird untersucht, ob und welche Vorteile dezentrale Marktplätze gegenüber zentralen Plattformen haben.  Zunächst wird dazu ein Marktüberblick bestehender DLT-Datenmarktprototypen in der Forschung mittels einer strukturierten Literaturrecherche gegeben und eine Taxonomie abgeleitet.  Anschließend werden Anforderungen von Anwendern und Betreibern an Datenmarktplätze mittels Requirements Engineering abgeleitet. Basierend auf den Anforderungen werden drei Systemarchitekturen aufgestellt und diese anhand der Transaktionskosten,  Anforderungserfüllung sowie Kaufprozesse miteinander verglichen.  Mittels eines Hyperledger-Caliper-Lasttests wird die dezentrale Hyperledger Fabric abschließend auf Lauffähigkeit in Umgebungen mit niedriger Systemleistung untersucht.  Resümierend kommt diese Arbeit zu dem Schluss, dass DLT eine valide Trägertechnologie für Datenmarktplätze ist und die DLT-Datenmarktplätze in den Vergleichskategorien besser abschneiden als die zentrale Vergleichsarchitektur.  Durch Analysedienstleistungen können zudem neue Mehrwerte gegenüber den derzeitigen Datensilos geschaffen werden.
\vspace{1cm}
\paragraph{Schlüsselwörter:} DLT, Blockchain, Datenmarkt, Ethereum, Hyperledger Fabric


%Inhaltsverzeichnis
\addcontentsline{toc}{chapter}{Gliederung}
\tableofcontents
\newpage

%Abbildungsverzeichnis, bei Bedarf
%\addcontentsline{toc}{chapter}{Abbildungsverzeichnis}
\listoffigures
\newpage




   
%Symbolverzeichnis, bei Bedarf
%\addcontentsline{toc}{section}{Symbolverzeichnis}
%{\noindent\Large\bfseries Symbolverzeichnis}
%\newpage

%Tabellenverzeichnis, bei Bedarf
%\addcontentsline{toc}{chapter}{Tabellenverzeichnis}
\listoftables
\newpage
\clearpage

%Abkürzungsverzeichnis, bei Beadar
\addchap{Abkürzungsverzeichnis}
\begin{acronym}[Abkürzungen] %Wort in eckigen Klammern bezieht sich auf längste Abkürzung und dient dem Abstand zwischen Abkürzungen und langem Wort
\acro{api}[API]{Application Programming Interface}
\acro{abi}[ABI]{Application Programming Interface}
\acro{bpmn}[BPMN]{Business Process Model and Notation}
\acro{btc}[BTC]{Bitcoin (Tokenwährung)}
\acro{dag}[DAG]{Directed Acyclic Graph}
\acro{dlt}[DLT]{Distributed Ledger Technology}
\acro{eip}[EIP]{Ethereum Improvement Proposal}
\acro{erc}[ERC]{Ethereum Request for Comments}
\acro{eth}[ETH]{Ethereum (Tokenwährung)}
\acro{ico}[ICO]{Initial Coin Offering}
\acro{iin}[IIN]{Interbank Information Network}
\acro{iot}[IoT]{Internet of Things}
\acro{mpc}[MPC]{Multi-Party Computation}
\acro{msp}[MSP]{Membership Service Provider}
\acro{osn}[OSN]{Ordering Service Node}
\acro{tak}[TAK]{Transaktionskosten}
\acro{p2pkh}[P2PKH]{Pay-to-Public-Key-Hash}
\acro{p2sh}[P2SH]{Pay-to-Script-Hash}
\acro{s2des}[S2DES]{Smart Sensor-based Digital Ecosystem Services (Forschungsprojekt)}
\acro{sgx}[SGX]{Software Guard Extensions}
\acro{soc}[SoC]{System-on-a-Chip}
\acro{uc}[UC]{Use Case}
\acro{uml}[UML]{Unified Modelling Language}
\end{acronym}

\newpage

%Aktuelle römische Seitenzahl speichern. um später weiterzuführen
\setcounter{savepage}{\arabic{page}}

\pagenumbering{arabic}

%Include each chapter
\section{Einleitung}
Anstelle \textit{Ihrer} Einleitung steht hier zunächst eine kurze Einleitung in die Benutzung dieser Formatvorlage.

\subsection{PDF-Erstellung und eigener Inhalt}
Zur Übersetzung dieser Vorlage müssen Sie biber verwenden, da Biblatex für die Quellenverwaltung verwendet wurde. \\
Für Ihre eigenen Kapitel können Sie eine neue .tex-Datei im Unterordner Kapitel anlegen und diese in der Hauptdatei referenzieren. In Texmaker bietet es sich an die Vorlage\_ Latex als Masterdatei zu definieren. So kann auch in den Unterdateien die Funktion Schnelles Übersetzen verwendet werden.
Eigene Quellenverzeichnisse können Sie über einen Verweis in der Präambel einbinden.


\subsection{Zitationsbeispiele}
Zitate können Sie über die Bordmittel simpel einbinden, da diese automatisch an den gewählten Zitationsstil angepasst werden \parencite[5-8]{bergener_wissenschaftliches_2019}.
Auch Zitate im Fließtext \textcite{bergener_wissenschaftliches_2019} lassen sich abbilden. Die Grundlagen zur neuesten APA-Zitationsweise können Sie im Originalwerk nachlesen \parencite{apa_2019}.

\subsection{Tabellen und Bilder}

\begin{figure}[htbp] 
  \centering
     \includegraphics{Bilder/buchmitkopf.png}
  \caption[Buch mit Kopf]{Buch mit Kopf \parencite[1-2]{bergener_wissenschaftliches_2019}}
  \label{fig:Bild1}
\end{figure}

\begin{table}[htbp]
%Table-Umgebung erstellen
\centering
%Tabelle erstellen
\begin{tabular}{|p{0.3\textwidth}|p{.3\textwidth}|p{.3\textwidth}|}
\hline
\textbf{Überschrift} & \textbf{Überschrift} & \textbf{Überschrift} \\ \hline
Zelltext             & Zelltext             &   Zelltext             \\ \hline
Zelltext             & Zelltext             & Zelltext             \\ \hline
\end{tabular}
\caption[Beschreibung der Tabelle]{Beschreibung der Tabelle \parencite[1-2]{bergener_wissenschaftliches_2019}}
\label{tab:beispieltabelle}
\end{table}
Im Gegensatz zu Word kann in Latex der Text für Tabellenverzeichnis und Tabellenunterschrift unterschiedlich definiert werden. Es bietet sich daher an die Zitation zur Tabelle direkt in der Unterschrift zu tätigen, statt dies im Fließtext nachzuholen. Ein Beispiel für die Verwendung von Tabellen sehen Sie mit Tabelle \ref{tab:beispieltabelle}. Verschiedene Tabellengeneratoren im Internet erleichtern die Verwendung von Tabellen enorm.











\newpage
\pagenumbering{Roman}
\setcounter{page}{\thesavepage}
%Anhang, bei Bedarf
%\newpage
\appendix
\renewcommand*\chapterformat{Anhang \thechapter:\enskip}
%\include{lit-anhang}
%\include{anf-anhang}
%\include{arch-anhang}

\clearpage
% Literaturverzeichnis
\singlespacing
\printbibliography[title=Literaturverzeichnis]
\onehalfspacing

\newpage
\clearpage


%\addcontentsline{toc}{chapter}{Ehrenwörtliche Erklärung}
\addchap{Ehrenwörtliche Erklärung}
Ich versichere, dass ich die Masterarbeit selbstständig verfasst und keine anderen als die angegebenen Quellen und Hilfsmittel benutzt habe.

Darüber hinaus versichere ich, dass die elektronische Version der Masterarbeit mit der gedruckten Version übereinstimmt.


\vspace{4cm}

\hspace{2cm} Ort, Datum \hfill Unterschrift \hspace{2cm}

\end{document}