\addchap{Abstract}
In dieser Arbeit wird untersucht, ob die Distributed-Ledger-Technologie (DLT) sich zur Verwendung als Trägertechnologie für einen Datenmarkt eignet und so domänenübergreifende Datennutzung ermöglicht werden kann.  Ebenso wird untersucht, ob und welche Vorteile dezentrale Marktplätze gegenüber zentralen Plattformen haben.  Zunächst wird dazu ein Marktüberblick bestehender DLT-Datenmarktprototypen in der Forschung mittels einer strukturierten Literaturrecherche gegeben und eine Taxonomie abgeleitet.  Anschließend werden Anforderungen von Anwendern und Betreibern an Datenmarktplätze mittels Requirements Engineering abgeleitet. Basierend auf den Anforderungen werden drei Systemarchitekturen aufgestellt und diese anhand der Transaktionskosten,  Anforderungserfüllung sowie Kaufprozesse miteinander verglichen.  Mittels eines Hyperledger-Caliper-Lasttests wird die dezentrale Hyperledger Fabric abschließend auf Lauffähigkeit in Umgebungen mit niedriger Systemleistung untersucht.  Resümierend kommt diese Arbeit zu dem Schluss, dass DLT eine valide Trägertechnologie für Datenmarktplätze ist und die DLT-Datenmarktplätze in den Vergleichskategorien besser abschneiden als die zentrale Vergleichsarchitektur.  Durch Analysedienstleistungen können zudem neue Mehrwerte gegenüber den derzeitigen Datensilos geschaffen werden.
\vspace{1cm}
\paragraph{Schlüsselwörter:} DLT, Blockchain, Datenmarkt, Ethereum, Hyperledger Fabric
